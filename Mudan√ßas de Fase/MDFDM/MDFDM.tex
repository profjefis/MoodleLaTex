\documentclass[12pt]{article}
\usepackage{moodle}
\usepackage{fontspec}
\usepackage[brazilian]{babel}
%\usepackage[utf8]{inputenc}
%\usepackage[T1]{fontenc}
\usepackage{graphicx}
\ifwindows\imagemagickcommand{magick convert}
\else\imagemagickcommand{convert}
\fi

\begin{document}
	\begin{quiz}{Mudanças de Fase - Discursiva - Médio}
		\begin{essay}[points=2,penalty=0,response format=file,attachments allowed=1,attachments required=1]{MDF01DM-210}
			Uma das extremidades de uma barra metálica isolada é mantida a 100 ºC, e a outra extremidade é mantida a 0 ºC por uma mistura de gelo e água. A barra tem 60,0 cm de comprimento e uma seção reta com área igual a 1,5 $cm^{2}$. O calor conduzido pela barra produz a fusão de 9,0 g de gelo em 10 minutos. Calcule a condutividade térmica do metal em W/m.K.
			Dado: calor latente de fusão da água = $3,5\times10^{5}$ J/kg.				
		\end{essay}	
		\begin{essay}[points=2,penalty=0,response format=file,attachments allowed=1,attachments required=1]{MDF02DM-225/20}
			Uma barra de alumínio (K = 0,5 cal/s.cm.ºC) está em contato, numa extremidade, com gelo em fusão e, na outra, com vapor de água em ebulição sob pressão normal. Seu comprimento é 25 cm, e a seção transversal tem 5,0 $cm^{2}$ de área. Sendo a barra isolada lateralmente e dados os calores latentes de fusão do gelo e de vaporização da água ($L_{F}$ = 80 cal/g ; $L_{V}$ = 540 cal/g), determine:\\ 
			(a)	a massa do gelo, em gramas, que se funde em meia hora;\\
			(b)	a temperatura, em ºC, em uma seção da barra a 5,0 cm da extremidade fria.			
		\end{essay}
		\begin{essay}[points=2,penalty=0,response format=file,attachments allowed=1,attachments required=1]{MDF03DM-50/2,4}
			\textbf{(UFG GO/2014)} O corpo humano consegue adaptar-se a diferentes temperaturas externas, mantendo sua temperatura aproximadamente constante em 37 ºC por meio da produção de energia por processos metabólicos e trocas de calor com o ambiente. Em uma situação típica, em que um indivíduo esteja em repouso em um ambiente a 25 ºC, ele libera calor para o ambiente por condução térmica a uma taxa de 15 J/s e por evaporação de água por meio da pele a uma taxa de 60 kJ/hora.\\			
			Considerando o exposto, calcule:\\
			Dados:\\
			Calor latente de evaporação da água à 37 ºC: L = 2400 kJ/kg.\\
			Densidade da água: d = 1 kg/litro.\\				
			a)	a quantidade de água, em ml, que o indivíduo deve ingerir para compensar a perda por evaporação em duas horas.\\
			b)	a espessura média da pele do indivíduo, em mm, considerando a área total da superfície da sua pele igual a 1,5 $m^{2}$ e a condutibilidade térmica (k) da mesma igual a $2\times10^{3}$ W/m.ºC.				
		\end{essay}
		\begin{essay}[points=2,penalty=0,response format=file,attachments allowed=1,attachments required=1]{MDF04DM-80/2,0}
			\textbf{(UFG GO/2013)} Uma caixa de isopor em forma de paralelepípedo de dimensões 0,4 x 0,6 x 0,4 m contém 9 kg de gelo em equilíbrio térmico com água. Esse sistema é fechado e mantido em uma sala cuja temperatura ambiente é de 30º C. Tendo em vista que o gelo é completamente derretido após um intervalo de 10 horas, calcule:\\
			Dados:\\
			1 cal $\approx$ 4,0 J.\\
			calor latente de fusão do gelo = 80 cal/g.\\			
			a) o fluxo de calor, em watt, que o conteúdo da caixa de isopor recebe até derreter o gelo;\\
			b) a espessura da caixa de isopor em centímetros. Utilize o coeficiente de transmissão de calor do isopor $4,0\times10^{–2}$ W/m.ºC.								
		\end{essay}
		\begin{essay}[points=2,penalty=0,response format=file,attachments allowed=1,attachments required=1]{MDF05DM-6,73}
			\textbf{(UDESC-SC)} Um sistema para aquecer água, usando energia solar, é instalado em uma casa para fornecer 400 L de água quente a 60 ºC durante um dia. A água é fornecida para casa a 15 ºC e a potência média por unidade de área dos raios solares é 130 $W/m^{2}$. Calcule a área da superfície dos painéis solares necessários em $m^{2}$.\\
			Dados:\\
			Calor específico da água 4200 J/Kg.ºC.\\
			Densidade da água 1 kg/L.											
		\end{essay}										
	\end{quiz}
\end{document}